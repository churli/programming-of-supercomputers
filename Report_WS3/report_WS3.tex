% Report
\documentclass{article}

% Here set the various packages
% Packages to load
\usepackage[english]{babel}
% %%% Support some german text
% \usepackage{ngerman}
% \usepackage[latin1]{inputenc}   % für Umlaute
%%%
% \usepackage[utf8]{inputenc}
\usepackage[T1]{fontenc}
\usepackage{microtype}

%%%
% \usepackage[inline]{enumitem} % Required for the "description" list.

%%% Fix for not hyperlinking citations
\makeatletter
\let\NAT@parse\undefined
\makeatother
\usepackage{hyperref}
% 
\usepackage{cite}
% \ifx\pdfoutput\undefined
% 	\usepackage{graphicx}
% \else
% 	\usepackage[pdftex]{graphicx}
% \fi
\usepackage{graphicx}
\graphicspath{{Figures/}}
\usepackage{amsmath}
% \interdisplaylinepenalty=2500

% Shading of questions. Use the "shaded" environment or the "\hl{}" command.
\usepackage{framed}
% \usepackage[dvipsnames]{color}
\usepackage[svgnames]{xcolor}
\usepackage{soul}
% Nice colours: Gainsboro, LightGoldenrod, LightSteelBlue
% furter ref: https://www.latextemplates.com/svgnames-colors
\definecolor{shadecolor}{named}{Gainsboro}
\sethlcolor{Gainsboro}

%%% Todo margin notes (enable/disable)
\usepackage{todonotes}
% \usepackage[disable]{todonotes}
%%%

%%% For more flexible enumerate environments
\usepackage{enumitem}

%%% For allowing verbatim text in captions
\usepackage{cprotect}
% Usage: \cprotect\caption{blablabla \verb!verbatim text!}

%eof

%%%

\title{Programming of Supercomputers\\Worksheet 3}
\author{
	\begin{tabular}{rl}
		Oleksandr Voloshyn& \texttt{<o.voloshyn@tum.de>}\\ 
		Qunsheng Huang& \texttt{<keefe.huang@tum.de>}\\ 
		Tommaso Bianucci& \texttt{<bianucci@in.tum.de>}
	\end{tabular}
}
\date{\today}

\begin{document}

\maketitle
\renewcommand{\abstractname}{Group members's contributions}
\begin{abstract}
	\begin{center}
		\begin{tabular}{rl}
		% Here write the contributions of the members of the group
		Oleksandr Voloshyn:& worked on 1, 3\\
		Qunsheng Huang:& worked on 1, 3\\
		Tommaso Bianucci:& worked on 2
		\end{tabular}
	\end{center}
\end{abstract}

\section{Setting a Baseline}
\subsection{Required submission files}
\begin{enumerate}
	\item \hl{The updated Load-Leveler batch script.}

		\verb!Data/Baseline/updated_job.ll!

	\item \hl{The performance plots and description in the report.}

		Figures \ref{fig:IO}, \ref{fig:setup}, \ref{fig:total}, \ref{fig:varying_size}, and\ref{fig:varying_domain} show the various performance plots for the baseline. A Vampir baseline plot is shown in Fig. \ref{fig:vampir_baseline}.

\end{enumerate}

\subsection{Questions}
\begin{enumerate}
	\item \hl{Briefly describe the Gaussian Elimination and the provided implementation.}

	The Gaussian Elimination (GE) is a program that converts a dense matrix into an upper-right triangular matrix. Typically, this is done via a recursive algorithm that sets entire columns of the matrix to zero by. A naive implementation is as follows:
\begin{enumerate}
\item Choosing a pivot. This is typically the value in the diagonal. In the current implementation, no shifting of rows is done and this ignored for the naive implementation.
\item The factor $\frac{a_{ij}}{a_{ii}}$ is calculated for each row in the column that is below the diagonal, such that $i > j$ if the topmost leftmost matrix entry is set as $a_{00}$.
\item All rows in the column below the diagonal are set to zero by linear operations using calculated factor.
\end{enumerate}

The provided implementation implements a few changes to simplify the calculations done in parallel.

Firstly, the full matrix is read in by process 0. The root process then performs some ordering of the matrix data and sends rows of the matrix to each process. Asserts are put in place to ensure that:
\begin{itemize}
\item The number of rows are equal to the number of columns
\item The number of processes must fully divide the number of rows/columns. $\#Rows%\#Processes==0$
\end{itemize}

The algorithm then works as follows:
\begin{enumerate}
\item Determine pivot entry. Each loop in the GE algorithm sets all values in a column below the diagonal to zero via linear operations. This is a purely sequential task, there is always only one pivot entry at one point in time and the current pivot is always on the diagonal entry of the current worked row. We refer to the process in charge of updating the rows containing the current pivot entry as the \textit{current process}. 

An assert is put in place to ensure that this pivot value not zero. 
\item Given a particular pivot, the entire row, including the RHS value, is first normalized by the pivot entry $norma_{ij} = \frac{a_{ij}}{a_{ii}} \forall j \in #Cols}$ and $normrhs_{i} = \frac{rhs_{i}}{a_{ii}}
$. 
\item Then, an elimation step is performed in the current process, where all values directly below the current pivot entry are set to zero by linear operations. Since the pivot entry is always set to 1, the linear operation reduces each element in the given row per ${a_{kj} -= norma_{ij} * a_{kj} \forall #Cols > j > i, #Rows > k > i}$ where i is the row of the pivot entry. 
\item The pivot then shifts to the diagonal entry in the next row and the process is repeated until the current process has looped through all rows it controls.
\item All normalized values in the matrix and in the rhs are stored in given buffers and sent to each process with a higher rank than the current process.
\item Each process then performs the eliminations steps using the provided information.
\item The pivot then shifts a row controlled by a new process and the normalization and elimination steps are repeated until all processes are complete.
\item The solution data is aggregated to process 0 and is stored as necessary.
\end{enumerate}

Effectively, information is sent to each processor in a sequential, cascading manner. The Gaussian Elimination (GE) code presents a major challenge in load-balancing---as the algorithm progresses, the amount of work available for each process decreases. When the last few rows are undergoing GE, all but one process is still performing work. 

We ran vampir on the baseline process to observe the blocking \verb!MPI_Send! and \verb!MPI_Recv! functions. This can be seen in Fig. \ref{fig:vampir_baseline}.

 \begin{figure}[h] % h=here, t=top, b=bottom, p=(extra)page, !=force
 	\begin{center}
  		\includegraphics[width=.7\linewidth]{Baseline/vampir_baseline.png} % It searches in the Figures/ folder!
  		\caption{Vampir Benchmark}
  		\label{fig:vampir_baseline}
  	\end{center}
 \end{figure}

	\item \hl{How is data distributed among the processes?}

	Each process takes a number of rows of the matrix equal to $\frac{row}{\#processes}$. There are checks in place such that the number of processes must be able to fully divide the total number of rows.

	\item \hl{Explain the changes applied to the provided Load-Leveler batch script.}

	One of the main difficulties is getting a good average value for the times. As such, the main way to reduce variance in results is to run each instance multiple times. In our case, we ran the code 5 times for each combination of MPI processes and domain size.

	\item \hl{What were the challenges in getting an accurate baseline time for Gaussian Elimination.}

	The main challenges were to determine what was important when determining a baseline. The Gaussian When improving the aglorithm, the main focus will be to reduce the amount of sequential communication---so that each process is able to receive updated information with as little wait time as possible. Thus, our chosen baseline should show improvement when the communication algorithm is improved. The following show the respective baselines chosen for each of the measured times:
\begin{enumerate}
	\item I/O Time

	I/O is only performed on the main process or process 0. Thus, it only makes sense to note this from process 0. Fig. shows the average time for IO operations in process 0 for the various domain sizes. Since the IO operation takes place completely sequentially, the number of processes does not affect the times. We see the average IO times for each domain size in the Fig. \ref{fig:IO}.
	
	
% % Figure example
 \begin{figure}[h] % h=here, t=top, b=bottom, p=(extra)page, !=force
 	\begin{center}
  		\includegraphics[width=.7\linewidth]{Baseline/io_baseline.png} % It searches in the Figures/ folder!
  		\caption{IO Time vs. Domain Size}
  		\label{fig:IO}
  	\end{center}
 \end{figure}

	\item Setup Time

	The setup time measures the amount of time each process waits before they fully receive their local matrix rows and local rows of the RHS vector. Since the sending process from process 0 is sequential, process 0 uses blocking communication to send data to process 1, 2, 3 ... etc, some processes with higher rank will wait for a longer time. Fig. \ref{fig:setup} shows the \textit{longest} wait time of all processes. 
	
	We see that keeping the domain size constant and increasing the number of processes, shown in Fig. \ref{fig:setup}(a), results in relatively constant setup times. This only changes for the smaller domain sizes of 1024, 512 and 64 with a larger number of processes. This likely is due to the communication overhead beginning to dominate.
	
 We also see that the setup time increases with the total domain size---which make sense as a larger domain necesitates more data sent from the root process to each other process. This can be seen for any number of processes, as shown in Fig. \ref{fig:setup}(b). Additionally, we see again the poor performance for smaller domain sizes with a high number of processes.
	
	
	\begin{figure}[h] % h=here, t=top, b=bottom, p=(extra)page, !=force
		\hspace*{-0.25\linewidth}\begin{tabular}{cc}
			\includegraphics[width=.8\linewidth]{Baseline/setup_multdomain_baseline.png} & \includegraphics[width=.8\linewidth]{Baseline/setup_multproc_baseline.png} \\
		    (a) Fixed Domain Size Varying \#Processes & (b) Fixed \#Processes Varying Domain Size\\[6pt]\\
		\end{tabular}
		\caption{Setup Time, Haswell vs. Sandy Bridge.}
		\label{fig:setup}
	\end{figure}
	
	\item Compute Time
	
	The compute time measures the amount of time between the start of the gaussian elimination steps to the end of the gaussian elimination steps. This includes the MPI time. Therefore, to determine the actual compute time, we subtract the MPI time from the compute time and average the results across the multiple processes. This is because the overall computational work for each domain size should remain relatively constant---regardless of the number of processes used. If an algorithmic improvement occurs in the code, this should improve this benchmark.
	
	\item MPI Time
	
	The MPI time sums up the total amount of time spent during communication. We hope to decrease this value by improving the communication and computation overlap. The MPI time increases with increased domain size as well and generally decreases with an increased number of nodes (as the amount sent to each node decreases). However, for very small domain sizes, we see the cost of communication overtake the benefits of multiple nodes. This is seen for domain sizes of 64x64 with 64 processes.
	
	\item Total Time
	
	The total time indicates the amount of time required to fully execute the GE algorithm. We see in Fig. \ref{fig:total}(a), total time generally decreases with an increased number of processes (if the domain size is sufficiently large that communication costs do not outweight the benefits of additional processes performing computation). However, the more interesting plot is Fig. \ref{fig:total}(b), where we can identify the "ideal" number of processes for best performance by tracing the line with the lowest time at each domain size. It is in this graph that we can see when it becomes beneficial to introduce more processes for performing the GE.
 	
 	
 	\begin{figure}[h] % h=here, t=top, b=bottom, p=(extra)page, !=force
		\hspace*{-0.25\linewidth}\begin{tabular}{cc}
			\includegraphics[width=.8\linewidth]{Baseline/total_multdomain_baseline.png} & \includegraphics[width=.8\linewidth]{Baseline/total_multproc_baseline.png} \\
		    (a) Fixed Domain Size Varying \#Processes & (b) Fixed \#Processes Varying Domain Size\\[6pt]\\
		\end{tabular}
		\caption{Total Time, Haswell vs. Sandy Bridge.}
		\label{fig:total}
	\end{figure}
	\end{enumerate}

	\item \hl{Describe the compute and MPI times scalability with fixed process counts and varying size of input files for the Sandy Bridge and Haswell nodes. Did you observe any differences?}
	
	
	\begin{figure}[h] % h=here, t=top, b=bottom, p=(extra)page, !=force
		\hspace*{-0.25\linewidth}\begin{tabular}{cc}
			\includegraphics[width=.8\linewidth]{Baseline/compute_multproc_baseline.png} & \includegraphics[width=.8\linewidth]{Baseline/mpi_multproc_baseline.png} \\
		    (a) Compute Time vs. Domain Size & (b) MPI Time vs. Domain Size\\[6pt]\\
		\end{tabular}
		\caption{Fixed Process Counts \& Varying Size.}
		\label{fig:varying_size}
	\end{figure}
	
	With fixed process count and a varying domain size, we see that the compute time increases with domain size, as shown in Fig \ref{fig:varying_size}(a). This trend was observed in Fig.\ref{fig:total}(b) and is not surprising. When comparing the performance of the Haswell and Sandy Bridge architectures, we see that Haswell nodes perform better for a majority of the time. Haswell nodes generally have shorter compute times when compared with Sandy Bridge when the domain size is between 512 and 4096. However, the Sandy Bridge nodes have shorted compute times at small domain sizes or very large domain sizes with a high number of processes.

	With a fixed domain size and a varying process count, we see, unsurprisingly, in Fig. \ref{fig:varying_size}(b) that the MPI times increase with domain size as more values need to be communicated across processes. We also see that Haswell nodes have shorter MPI times compared to Sandy Bridge nodes for process counts of 8, 16 \& 32 when the domain size is below 4096. When the domain size is increased beyond 4096, Sandy Bridge nodes have shorter MPI times for all process counts other than 8. Additionally, we note that the Sandy Bridge nodes have shorter MPI times for almost any domain size when the process count is increased to 64.
	
	Overall, this seems to indicate that, within these constraints, the Haswell nodes will perform slightly better. However, these results also imply that the Sandy Bridge nodes may scale better with a larger domain size and an increased process count (assuming that the chosen process count is suitable for the given domain size). 

	\item \hl{Describe the compute and MPI times scalability with fixed input sets and varying process counts for the Sandy Bridge and Haswell nodes. Did you observe any differences?}

	\begin{figure}[h] % h=here, t=top, b=bottom, p=(extra)page, !=force
		\hspace*{-0.25\linewidth}\begin{tabular}{cc}
			\includegraphics[width=.8\linewidth]{Baseline/compute_multdomain_baseline.png} & \includegraphics[width=.8\linewidth]{Baseline/mpi_multdomain_baseline.png} \\
			(a) Compute Time vs. \#Processors & MPI Time vs. \#Processors\\[6pt]
		\end{tabular}
		\caption{Fixed Size \& Varying Process Counts.}
		\label{fig:varying_domain}
	\end{figure}


With fixed domain size and increasing the process count, we see that the compute time decreases with process count. This is not surprising because each process works on a smaller segment of the full matrix. We see in Fig. \ref{fig:varying_domain} that this remains true for any domain size. For small domain sizes, such as 64x64, the compute time is completely negligible in comparison to the communication time. For larger domain sizes, the compute time becomes much more significant. We see that the Haswell nodes has generally shorter compute times for any processor count. However, the Sandy Bridge nodes still seem to scale better with a larger number of processes at a large domain size.

We expect to see a longer MPI time as the number of processes increase. However, this does not seem to be the case across the board. This is true for the smaller domain sizes, but not the larger domain sizes, as seen in Fig \ref{fig:varying_domain}. This is somewhat surprising as we would expect the cost of comnunication to increase with increased processes. We attribute this observed effect to the sequential nature of the MPI passing---the first process waits for all other processes to finish computing before it returns. As a result, if we successfully overlap compute time with MPI time, we will see a decrease in the time spent waiting for blocking MPI calls.

We see that the Haswell architecture performs better overall, with the Sandy Bridge nodes performing slightly better, with regards to MPI time, for large domains with larger process counts.
\end{enumerate}

% % Figure example
% \begin{figure}[p] % h=here, t=top, b=bottom, p=(extra)page, !=force
%  	\begin{center}
%  		\includegraphics[width=.9\linewidth]{figure.png} % It searches in the Figures/ folder!
%  		\caption{Caption text}
%  		\label{fig:figureLabelName}
%  	\end{center}
% \end{figure}

\subsection{Questions}
\begin{enumerate}
\item Briefly describe the Gaussian Elimination and the provided implementation.

The Gaussian Elimination (GE) is a program that converts a dense matrix into an upper-right triangular matrix. This is done via a recursive algorithm that sets entire columns of the matrix to zero by:
\begin{enumerate}
\item Choosing a pivot. This is typically the value in the diagonal. Pivoting may be performed to shift the row with the largest value in that column into the correct position
\item The factor $\frac{a_{ij}}{a_{ii}}$ is calculated for each row in the column that is below the diagonal, such that $i > j$ if the topmost leftmost matrix entry is set as $a_{00}$.
\item All rows in the column below the diagonal are set to zero using the calculated factor
\end{enumerate}

The provided implementation uses segments the matrix into rows and provides a set of rows to each process. The value of pivots are sent to each processor in a sequential, cascading manner. Each row then calculates the factor mentioned in the algorithm and then performs the linear operations to set the correct column values to 0.

Gaussian Elmination code presents a major challenge in load-balancing---as the algorithm progresses, the amount of work available for each process decreases. When the last few rows are undergoing GE, all but one process is still performing work. 

\item How is data distributed among the processes?

Each process takes a number of rows of the matrix equal to $\frac{row}/{#processes}$. There are checks in place such that the number of processes must be able to fully divide the total number of rows.

\item Explain the changes applied to the provided Load-Leveler batch script.

One of the main difficulties is getting a good average value for the times. As such, the main way to reduce variance in results is to run each instance multiple times. In our case, we ran the code 5 times for each combination of MPI processes and domain size.

\item What were the challenges in getting an accurate baseline time for Gaussian Elimination.

The main challenges were to determine what was important when determining a baseline. The Gaussian When improving the aglorithm, the main focus will be to reduce the amount of sequential communication---so that each process is able to receive updated information with as little wait time as possible. Thus, our chosen baseline should show improvement when the communication algorithm is improved. The following show the respective baselines chosen for each of the measured times:
\begin{enumerate}
	\item I/O Time

	I/O is only performed on the main process or process 0. Thus, it only makes sense to note this from process 0. Fig. shows the average time for IO operations in process 0 for the various domain sizes. Since the IO operation takes place completely sequentially, the number of processes does not affect the times. 

	\item Setup Time

	The setup time measures the amount of time each process waits before they fully receive their local matrix rows and local rows of the RHS vector. Since the sending process from process 0 is sequential---process 0 uses blocking communication to send data to process 1, 2, 3 ... etc, some processes with higher rank will wait for longer. Improvement will then be indicated by the \textit{longest} wait time of all processes. The provided data shows either the longest wait time for each process
	
	\item Compute Time
	
	The compute time measures the amount of time between the start of the gaussian elimination steps to the end of the gaussian elimination steps. This includes the MPI time. Therefore, to determine the actual compute time, we subtract the MPI time from the compute time. The compute time is extremely small for low values of the domain but grows significantly for larger domains. The compute time always decreases with increasing number of processes.
	
	\time MPI Time
	
	The MPI time sums up the total amount of time spent during communication. We hope to decrease this value by improving the communication and computation overlap. The MPI time increases with increased domain size as well and generally decreases with an increased number of nodes (as the amount sent to each node decreases). However, for very small domain sizes, we see the cost of communication overtake the benefits of multiple nodes. This is seen for domain sizes of 64x64 with 64 processes.
	

	\end{enumerate}


\item Describe the compute and MPI times scalability with fixed process counts and varying size of input
files for the Sandy Bridge and Haswell nodes. Did you observe any differences?

With fixed process count and increasing the domain size, we see that the compute time increases with domain size. This is not surprising as there is more time required for computation. We see in Fig. that this remains true for any number of processes. However, we see that while the Haswell nodes are faster at a 8 processes, the Sandy Bridge nodes are faster when the number of nodes is increased to 64. This seems to indicate that Sandy Bridge processors scale better with a larger number of processes.

We see that the MPI times follow a similar pattern---with a larger domain size, the amount of time in communication increases. This remains true at any number of processes. The shape of the graphs are very similar to that seen for the compute time. We also see similar behavior that the Sandy Bridge processors scale better with a larger number of processes.
 
\item Describe the compute and MPI times scalability with fixed input sets and varying process counts for
the Sandy Bridge and Haswell nodes. Did you observe any differences?

With fixed domain size and increasing the process count, we see that the compute time decreases with process count. This is not surprising because each process works on a smaller segment of the full matrix. We see in Fig. that this remains true for any domain size. For small domain sizes, such as 64x64, the compute time is completely negligible in comparison to the communication time. For larger domain sizes, the compute time becomes much more significant. We see that the Haswell nodes remain faster for any processor when the domain size is small. Once the domain size exceeds a certain threshold, the Sandy Bridge nodes are seen to scale better with a larger number of processes.

We see that the MPI times follow a similar pattern---with more processors, MPI time decreases. This is somewhat surprising as we would expect the cost of comnunication to increase with increased processes. We attribute this observed effect to the sequential nature of the MPI passing---the first process waits for all other processes to finish computing before it returns. As a result, by reducing overall computational time, we decrease the time spent waiting for blocking MPI calls. We see that the Sandy Bridge architecture performs better overall---at very small process counts it is, overall, slightly faster than the Haswell architecture. At larger domain sizes, the difference in MPI times very large, but shrinks with an increasing number of processors.

\end{enumerate}

\section{MPI Point-to-Point Communication}
\subsection{Required submission files}
\begin{enumerate}
	\item \hl{The updated \emph{gauss.c} file.}

		\verb!Data/MPI_P2P/gauss.c!

	\item \hl{The new performance plots and description in the report.}

		Figures \ref{fig:compute_multdomain_nb_baseline}, \ref{fig:compute_multproc_nb_baseline}, \ref{fig:mpi_multdomain_nb_baseline}, \ref{fig:mpi_multproc_nb_baseline}, \ref{fig:total_multdomain_nb_baseline} and \ref{fig:total_multproc_nb_baseline} show the various performance plots for the MPI P2P case. A Vampir plot is shown in Fig. \ref{fig:vampir_nonblocking}.

\end{enumerate}

\subsection{Questions}
\begin{enumerate}
	\item \hl{Which non-blocking operations were used? Justify your choice.}

	As this assignment was about point-to-point (P2P) communication, we restricted the choice of possible communication methods just to the P2P ones, avoiding collectives.

	The non-blocking P2P communication methods used are \verb!MPI_Isend()! and \verb!MPI_Irecv()!.

	\verb!MPI_Isend()! was used to avoid any waiting time on the sender in case the receiver has not reached the communication step yet. In this code the local computation can always proceed independently of the data being received on other processes and there is no risk of overwriting the send buffer.

	\verb!MPI_Irecv()! was instead used to try to anticipate the communication, e.g. by calling it during a previous loop iteration with respect to when the data is actually required.

	Not all the original blocking P2P communication calls were translated into their non-blocking counterparts: for example some \verb!MPI_Recv()! calls during the initial data distribution phase have been kept in their blocking version as there was no benefit in using the non-blocking version, i.e. no possibility to overlap the communication with any computation task.

	\item \hl{Was communication and computation overlap achieved? Use Vampir.}

	Not really. It is true that some degree of overlap whas achieved mainly by performing the communication for the following loop iteration while computing the current one. However the Vampir output --- see Fig.\ref{fig:vampir_nonblocking} --- confirms the theoretical understading of the gaussian elimination as an inherently sequential algorithm. In the figure we often see red patches within the green bands: these are because processes are waiting for information from other processes with lower rank in order to perform their calculations.

	\begin{figure}[h] % h=here, t=top, b=bottom, p=(extra)page, !=force
	\begin{center}
		\includegraphics[width=.8\linewidth]{MPI_p2p/vampir_nonblocking.png}
		\caption{Vampir output for non-blocking communication}
		\label{fig:vampir_nonblocking}
	\end{center}
	\end{figure}

	We also changed the communication pattern so that pivots are propagated after each row computation, instead of communicating after the computation entire local block. This has the benefit of allowing an earlier start of all the processes, however it does not really improve the cumulative time spent by the various processes waiting for data.

	So there is some overlap, but the inherent load imbalance of the algorithm and the sequential dependencies across the processes basically nullify any benefit coming from the computation-communication overlap.

	\item \hl{Was a speedup observed versus the baseline for the Sandy Bridge and Haswell nodes?}

	On average, the non-blocking version performed on par or slightly better than the baseline --- see Fig.\ref{fig:total_multdomain_nb_baseline} and \ref{fig:total_multproc_nb_baseline} --- however it does not constitute a significant speedup by any means.

	This is, again, due to the inherently unbalanced nature of Gaussian Elimination: each process needs the data computed by the previous ones and the further it proceeds in the computation, the more processes are left idle.

	\begin{figure}[p] % h=here, t=top, b=bottom, p=(extra)page, !=force
		\begin{tabular}{cc}
			\hspace*{-0.35\linewidth}\includegraphics[width=.85\linewidth]{MPI_p2p/compute_multdomain_haswell_nb_baseline.png} & \hspace*{-0.05\linewidth}\includegraphics[width=.85\linewidth]{MPI_p2p/compute_multdomain_sandy_nb_baseline.png} \\
			\hspace*{-0.45\linewidth}(a) Haswell & \hspace*{-0.15\linewidth}(b) Sandy Bridge\\[6pt]
		\end{tabular}
		\caption{Non-blocking: Compute Time vs. \#Processes.}
		\label{fig:compute_multdomain_nb_baseline}
	\end{figure}
	
	\begin{figure}[p] % h=here, t=top, b=bottom, p=(extra)page, !=force
		\begin{tabular}{cc}
			\hspace*{-0.35\linewidth}\includegraphics[width=.85\linewidth]{MPI_p2p/compute_multproc_haswell_nb_baseline.png} & \hspace*{-0.05\linewidth}\includegraphics[width=.85\linewidth]{MPI_p2p/compute_multproc_sandy_nb_baseline.png} \\
			\hspace*{-0.45\linewidth}(a) Haswell & \hspace*{-0.15\linewidth}(b) Sandy Bridge\\[6pt]
		\end{tabular}
		\caption{Non-blocking: Compute Time vs. Domain Size.}
		\label{fig:compute_multproc_nb_baseline}
	\end{figure}
	
	\begin{figure}[p] % h=here, t=top, b=bottom, p=(extra)page, !=force
		\begin{tabular}{cc}
			\hspace*{-0.35\linewidth}\includegraphics[width=.85\linewidth]{MPI_p2p/mpi_multdomain_haswell_nb_baseline.png} & \hspace*{-0.05\linewidth}\includegraphics[width=.85\linewidth]{MPI_p2p/mpi_multdomain_sandy_nb_baseline.png} \\
			\hspace*{-0.45\linewidth}(a) Haswell & \hspace*{-0.15\linewidth}(b) Sandy Bridge\\[6pt]
		\end{tabular}
		\caption{Non-blocking: MPI Time vs. \#Processes.}
		\label{fig:mpi_multdomain_nb_baseline}
	\end{figure}
	
	\begin{figure}[p] % h=here, t=top, b=bottom, p=(extra)page, !=force
		\begin{tabular}{cc}
			\hspace*{-0.35\linewidth}\includegraphics[width=.85\linewidth]{MPI_p2p/mpi_multproc_haswell_nb_baseline.png} & \hspace*{-0.05\linewidth}\includegraphics[width=.85\linewidth]{MPI_p2p/mpi_multproc_sandy_nb_baseline.png} \\
			\hspace*{-0.45\linewidth}(a) Haswell & \hspace*{-0.15\linewidth}(b) Sandy Bridge\\[6pt]
		\end{tabular}
		\caption{Non-blocking: MPI Time vs. Domain Size.}
		\label{fig:mpi_multproc_nb_baseline}
	\end{figure}
	
	\begin{figure}[p] % h=here, t=top, b=bottom, p=(extra)page, !=force
		\begin{tabular}{cc}
			\hspace*{-0.35\linewidth}\includegraphics[width=.85\linewidth]{MPI_p2p/total_multdomain_haswell_nb_baseline.png} & \hspace*{-0.05\linewidth}\includegraphics[width=.85\linewidth]{MPI_p2p/total_multdomain_sandy_nb_baseline.png} \\
			\hspace*{-0.45\linewidth}(a) Haswell & \hspace*{-0.15\linewidth}(b) Sandy Bridge\\[6pt]
		\end{tabular}
		\caption{Non-blocking: Total Time vs. \#Processes.}
		\label{fig:total_multdomain_nb_baseline}
	\end{figure}
	
	\begin{figure}[p] % h=here, t=top, b=bottom, p=(extra)page, !=force
		\begin{tabular}{cc}
			\hspace*{-0.35\linewidth}\includegraphics[width=.85\linewidth]{MPI_p2p/total_multproc_haswell_nb_baseline.png} & \hspace*{-0.05\linewidth}\includegraphics[width=.85\linewidth]{MPI_p2p/total_multproc_sandy_nb_baseline.png} \\
			\hspace*{-0.45\linewidth}(a) Haswell & \hspace*{-0.15\linewidth}(b) Sandy Bridge\\[6pt]
		\end{tabular}
		\caption{Non-blocking: Total Time vs. Domain Size.}
		\label{fig:total_multproc_nb_baseline}
	\end{figure}

\end{enumerate}

% % Figure example
% \begin{figure}[p] % h=here, t=top, b=bottom, p=(extra)page, !=force
%  	\begin{center}
%  		\includegraphics[width=.9\linewidth]{figure.png} % It searches in the Figures/ folder!
%  		\caption{Caption text}
%  		\label{fig:figureLabelName}
%  	\end{center}
% \end{figure}

\section{MPI One-Sided Communication}
\subsection{Required submission files}
\begin{enumerate}
  \item \hl{The updated \emph{gauss.c} file.}

    \verb!Data/path/to/file!

  \item \hl{The new performance plots and description in the report.}

	Please refer to the following Fig \ref{fig:compute_multdomain_os_baseline}, \ref{fig:compute_multproc_os_nb}, \ref{fig:mpi_multdomain_os_baseline}, \ref{fig:mpi_multproc_os_baseline}, \ref{fig:total_multdomain_os_baseline}, \ref{fig:total_multproc_os_baseline}, \ref{fig:compute_multdomain_os_nb}, \ref{fig:compute_multproc_os_nb}, \ref{fig:mpi_multdomain_os_nb}, \ref{fig:mpi_multproc_os_nb}, \ref{fig:total_multdomain_os_nb}, \ref{fig:total_multproc_os_nb}. The vampir plots for Sandy and Haswell are Fig. \ref{fig:vampir_sandy_one_sided} \& \ref{fig:vampir_sandy_one_sided}.

\end{enumerate}

\subsection{Questions}
\begin{enumerate}
  \item \hl{Which one-sided operations were used? Justify your choice.}

	We targeted the blocking \verb!MPI_Send! and \verb!MPI_Recv! functions passing pivot data for optimization for optimization.
	These were the most affected by the blocking operations and was the most likely to benefit from overlapping communication and computation.
    We decided to use the Post-Start-Wait-Complete methodology, which uses the following 
    MPI functions: \verb!MPI_Win_start!, \verb!MPI_Win_post!, \verb!MPI_Get! and \verb!MPI_Win_complete!. 
    This methodology allows the use of unique MPI communication groups---specifying the specific processes which are allowed to participate in a particular exposure/access epoch.
    This is useful for this use case as the communication of information only goes from processes with a lower rank number to processes with a higher rank number when distributing the pivot data.
    There is additionally a final aggregation phase where the computed information is passed back to process 0.
    However, the time associated with this phase is negligible in relation to the time spent communicating pivot data.
    We have decided to go with \verb!MPI_Get! function because it fits the purpose of distributing
    data from the process that had computed his part of pivoting to the rest.

  \item \hl{Was communication and computation overlap achieved? Use Vampir.}

    Yes it was achieved and it can be seen in Fig \ref{fig:vampir_haswell_one_sided} \& \ref{fig:vampir_sandy_one_sided}. 
    We can clearly see in both examples that there is computation occuring as certain processes are performing communication. 
    \begin{figure}[h] % h=here, t=top, b=bottom, p=(extra)page, !=force
	\begin{center}
			\includegraphics[width=.8\linewidth]{/one_sided/vampir_onesided_hw.png}
		\caption{Vampir output for One-Sided Communication, Haswell}
		\label{fig:vampir_haswell_one_sided}
	\end{center}
	\end{figure}
	
	    \begin{figure}[h] % h=here, t=top, b=bottom, p=(extra)page, !=force
	\begin{center}
			\includegraphics[width=.8\linewidth]{/one_sided/vampir_onesided_sb.jpg}
		\caption{Vampir output for One-Sided Communication, Sandy Bridge}
		\label{fig:vampir_sandy_one_sided}
	\end{center}
	\end{figure}

  \item \hl{Was a speedup observed versus the baseline for the Sandy Bridge and Haswell nodes?}

    Results performed on Haswell and Sandy Bridge for One-sided communication are almost the same.
  
    Despite the fact that overlap of the communication and computation was achieved, overall time
    was even worse than for baseline. 
    This was due the fact that processors with a higher rank number 
    had long computation times. Therefore, they become a bottleneck
    for this implementation. 
    This is expected since these portions will perform the Gaussian Elimination
    steps more times than the processes with lower ranks.
    
    Additionally, the Guassian Elimination algorithm is inherently sequential and hard to load balance
    properly---resulting in the observed phenomenon where certain processes perform far more compuation
    than others.
    
    We can view the results comparing the computational time and MPI time of the Baseline and the One-Sided communication below.
    
    	\begin{figure}[h] % h=here, t=top, b=bottom, p=(extra)page, !=force
		\hspace*{-0.25\linewidth}\begin{tabular}{cc}
			\includegraphics[width=.75\linewidth]{one_sided/compute_multdomain_haswell_os_baseline.png} & \includegraphics[width=.75\linewidth]{one_sided/compute_multdomain_sandy_os_baseline.png} \\
			(a) Haswell &  (b) Sandy Bridge\\[6pt]
		\end{tabular}
		\caption{Compute Time vs. \#Processes., One-Sided vs. Baseline}
		\label{fig:compute_multdomain_os_baseline}
	\end{figure}
	
	\begin{figure}[h] % h=here, t=top, b=bottom, p=(extra)page, !=force
		\hspace*{-0.25\linewidth}\begin{tabular}{cc}
			\includegraphics[width=.8\linewidth]{one_sided/compute_multproc_haswell_os_baseline.png} & \includegraphics[width=.8\linewidth]{one_sided/compute_multproc_sandy_os_baseline.png} \\
			(a) Haswell &  (b) Sandy Bridge\\[6pt]
		\end{tabular}
		\caption{Compute Time vs. Domain Size., One-Sided vs. Baseline}
		\label{fig:compute_multproc_os_baseline}
	\end{figure}
	
		\begin{figure}[h] % h=here, t=top, b=bottom, p=(extra)page, !=force
		\hspace*{-0.25\linewidth}\begin{tabular}{cc}
			\includegraphics[width=.8\linewidth]{one_sided/mpi_multdomain_haswell_os_baseline.png} & \includegraphics[width=.8\linewidth]{one_sided/mpi_multdomain_sandy_os_baseline.png} \\
			(a) Haswell &  (b) Sandy Bridge\\[6pt]
		\end{tabular}
		\caption{MPI Time vs. \#Processes., One-Sided vs. Baseline}
		\label{fig:mpi_multdomain_os_baseline}
	\end{figure}
	
		\begin{figure}[h] % h=here, t=top, b=bottom, p=(extra)page, !=force
		\hspace*{-0.25\linewidth}\begin{tabular}{cc}
			\includegraphics[width=.8\linewidth]{one_sided/mpi_multproc_haswell_os_baseline.png} & \includegraphics[width=.8\linewidth]{one_sided/mpi_multproc_sandy_os_baseline.png} \\
			(a) Haswell &  (b) Sandy Bridge\\[6pt]
		\end{tabular}
		\caption{MPI Time vs. Domain Size., One-Sided vs. Baseline}
		\label{fig:mpi_multproc_os_baseline}
	\end{figure}
	
			\begin{figure}[h] % h=here, t=top, b=bottom, p=(extra)page, !=force
		\hspace*{-0.25\linewidth}\begin{tabular}{cc}
			\includegraphics[width=.8\linewidth]{one_sided/total_multdomain_haswell_os_baseline.png} & \includegraphics[width=.8\linewidth]{one_sided/total_multdomain_sandy_os_baseline.png} \\
			(a) Haswell &  (b) Sandy Bridge\\[6pt]
		\end{tabular}
		\caption{Total Time vs. \#Processes., One-Sided vs. Baseline}
		\label{fig:total_multdomain_os_baseline}
	\end{figure}
	
		\begin{figure}[h] % h=here, t=top, b=bottom, p=(extra)page, !=force
		\hspace*{-0.25\linewidth}\begin{tabular}{cc}
			\includegraphics[width=.8\linewidth]{one_sided/total_multproc_haswell_os_baseline.png} & \includegraphics[width=.8\linewidth]{one_sided/total_multproc_sandy_os_baseline.png} \\
			(a) Haswell &  (b) Sandy Bridge\\[6pt]
		\end{tabular}
		\caption{Total Time vs. Domain Size., One-Sided vs. Baseline}
		\label{fig:total_multproc_os_baseline}
	\end{figure}

  \item \hl{Was a speedup observed versus the non-blocking version for the Sandy Bridge and Haswell nodes?}

	There was little speedup when comparing with the non-blocking version. Due to the inherent sequential nature of the communication. The non-blocking more consistently performed better than the one-sided communication. We can compare these results in the following plots.
	
	    	\begin{figure}[h] % h=here, t=top, b=bottom, p=(extra)page, !=force
		\hspace*{-0.25\linewidth}\begin{tabular}{cc}
			\includegraphics[width=.75\linewidth]{one_sided/compute_multdomain_haswell_os_nb.png} & \includegraphics[width=.75\linewidth]{one_sided/compute_multdomain_sandy_os_nb.png} \\
			(a) Haswell &  (b) Sandy Bridge\\[6pt]
		\end{tabular}
		\caption{Compute Time vs. \#Processes., One-Sided vs Non-Blocking}
		\label{fig:compute_multdomain_os_nb}
	\end{figure}
	
	\begin{figure}[h] % h=here, t=top, b=bottom, p=(extra)page, !=force
		\hspace*{-0.25\linewidth}\begin{tabular}{cc}
			\includegraphics[width=.8\linewidth]{one_sided/compute_multproc_haswell_os_nb.png} & \includegraphics[width=.8\linewidth]{one_sided/compute_multproc_sandy_os_nb.png} \\
			(a) Haswell &  (b) Sandy Bridge\\[6pt]
		\end{tabular}
		\caption{Compute Time vs. Domain Size., One-Sided vs Non-Blocking}
		\label{fig:compute_multproc_os_nb}
	\end{figure}
	
		\begin{figure}[h] % h=here, t=top, b=bottom, p=(extra)page, !=force
		\hspace*{-0.25\linewidth}\begin{tabular}{cc}
			\includegraphics[width=.8\linewidth]{one_sided/mpi_multdomain_haswell_os_nb.png} & \includegraphics[width=.8\linewidth]{one_sided/mpi_multdomain_sandy_os_nb.png} \\
			(a) Haswell &  (b) Sandy Bridge\\[6pt]
		\end{tabular}
		\caption{MPI Time vs. \#Processes., One-Sided vs Non-Blocking}
		\label{fig:mpi_multdomain_os_nb}
	\end{figure}
	
		\begin{figure}[h] % h=here, t=top, b=bottom, p=(extra)page, !=force
		\hspace*{-0.25\linewidth}\begin{tabular}{cc}
			\includegraphics[width=.8\linewidth]{one_sided/mpi_multproc_haswell_os_nb.png} & \includegraphics[width=.8\linewidth]{one_sided/mpi_multproc_sandy_os_nb.png} \\
			(a) Haswell &  (b) Sandy Bridge\\[6pt]
		\end{tabular}
		\caption{MPI Time vs. Domain Size., One-Sided vs Non-Blocking}
		\label{fig:mpi_multproc_os_nb}
	\end{figure}
	
			\begin{figure}[h] % h=here, t=top, b=bottom, p=(extra)page, !=force
		\hspace*{-0.25\linewidth}\begin{tabular}{cc}
			\includegraphics[width=.8\linewidth]{one_sided/total_multdomain_haswell_os_nb.png} & \includegraphics[width=.8\linewidth]{one_sided/total_multdomain_sandy_os_nb.png} \\
			(a) Haswell &  (b) Sandy Bridge\\[6pt]
		\end{tabular}
		\caption{Total Time vs. \#Processes.}
		\label{fig:total_multdomain_os_nb}
	\end{figure}
	
		\begin{figure}[h] % h=here, t=top, b=bottom, p=(extra)page, !=force
		\hspace*{-0.25\linewidth}\begin{tabular}{cc}
			\includegraphics[width=.8\linewidth]{one_sided/total_multproc_haswell_os_baseline.png} & \includegraphics[width=.8\linewidth]{one_sided/total_multproc_sandy_os_baseline.png} \\
			(a) Haswell &  (b) Sandy Bridge\\[6pt]
		\end{tabular}
		\caption{Total Time vs. Domain Size.}
		\label{fig:total_multproc_os_nb}
	\end{figure}
Theoretically, we should be able to optimize the communication by minimizing synchronization costs in the one-sided. However, this requires very situation specific tuning of parameters. One way we could perform this tuning is via chunking the rows in one-sided and changing the size of chunks to optimize the communication. However, this would be an extremely task-specific optimization.

\end{enumerate}
% % Figure example
% \begin{figure}[p] % h=here, t=top, b=bottom, p=(extra)page, !=force
%    \begin{center}
%      \includegraphics[width=.9\linewidth]{figure.png} % It searches in the Figures/ folder!
%      \caption{Caption text}
%      \label{fig:figureLabelName}
%    \end{center}
% \end{figure}


\end{document}

%eof
