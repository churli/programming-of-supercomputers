% Report
\documentclass{article}

% Here set the various packages
% Packages to load
\usepackage[english]{babel}
% %%% Support some german text
% \usepackage{ngerman}
% \usepackage[latin1]{inputenc}   % für Umlaute
%%%
% \usepackage[utf8]{inputenc}
\usepackage[T1]{fontenc}
\usepackage{microtype}

%%%
% \usepackage[inline]{enumitem} % Required for the "description" list.

%%% Fix for not hyperlinking citations
\makeatletter
\let\NAT@parse\undefined
\makeatother
\usepackage{hyperref}
% 
\usepackage{cite}
% \ifx\pdfoutput\undefined
% 	\usepackage{graphicx}
% \else
% 	\usepackage[pdftex]{graphicx}
% \fi
\usepackage{graphicx}
\graphicspath{{Figures/}}
\usepackage{amsmath}
% \interdisplaylinepenalty=2500

% Shading of questions. Use the "shaded" environment or the "\hl{}" command.
\usepackage{framed}
% \usepackage[dvipsnames]{color}
\usepackage[svgnames]{xcolor}
\usepackage{soul}
% Nice colours: Gainsboro, LightGoldenrod, LightSteelBlue
% furter ref: https://www.latextemplates.com/svgnames-colors
\definecolor{shadecolor}{named}{Gainsboro}
\sethlcolor{Gainsboro}

%%% Todo margin notes (enable/disable)
\usepackage{todonotes}
% \usepackage[disable]{todonotes}
%%%

%%% For more flexible enumerate environments
\usepackage{enumitem}

%%% For allowing verbatim text in captions
\usepackage{cprotect}
% Usage: \cprotect\caption{blablabla \verb!verbatim text!}

%eof



%%%

\title{Programming of Supercomputers\\Worksheet 1}
\author{Oleksandr Voloshyn\\ Qunsheng Huang\\ Tommaso Bianucci}
\date{\today}

\begin{document}

\maketitle
\renewcommand{\abstractname}{Group members's contributions}
\begin{abstract}
	% Here write the contributions of the members of the group
	Here briefly state the contributions of the different members of the group!
\end{abstract}

\section{Performance baseline} // Name of assignment 1
\subsection{GNU Profiler} // Name of sub-assignment
Lorem ipsum
\subsubsection{Questions}
\begin{enumerate}
\item{Which routines took 80\% or more of the execution time of the benchmark?
}
\begin{enumerate}
	\item{serial}
	\begin{enumerate}
		\item{CalcHourglassControlForElems(Domain\&, double*, double)}
		\item{EvalEOSForElems(Domain\&, double*, int, int*, int)}
		\item{CalcLagrangeElements(Domain\&)}
		\item{IntegrateStressForElems(Domain\&, double*, double*, double*, double*, int, int)}
		\item{CalcQForElems(Domain\&)}
		\item{Domain::Domain(int, int, int, int, int, int, int, int, int) }
		\item{ParseCommandLineOptions(int, char**, int, cmdLineOpts*) }
		\item{InitMeshDecomp(int, int, int*, int*, int*, int*)}
		\item{Domain::~Domain()}
		\item{VerifyAndWriteFinalOutput(double, Domain\&, int, int) }
	\end{enumerate}
	\item{OpenMP}
	\begin{enumerate}
		\item{CalcHourglassControlForElems(Domain\&, double*, double)}
		\item{IntegrateStressForElems(Domain\&, double*, double*, double*, double*, int, int)}
		\item{CalcLagrangeElements(Domain\&)}
		\item{CalcQForElems(Domain\&)}
		\item{Domain::Domain(int, int, int, int, int, int, int, int, int)}
		\item{ParseCommandLineOptions(int, char**, int, cmdLineOpts*)}
		\item{InitMeshDecomp(int, int, int*, int*, int*, int*)}
		\item{Domain::~Domain()}
		\item{VerifyAndWriteFinalOutput(double, Domain\&, int, int)}		
	\end{enumerate}
	\item{MPI}
	\begin{enumerate}
		\item{LagrangeNodal(Domain\&)}
		\item{EvalEOSForElems(Domain\&, double*, int, int*, int)}
		\item{CalcKinematicsForElems(Domain\&, double, int)}
		\item{CalcMonotonicQGradientsForElems(Domain\&)}
		\item{CalcMonotonicQRegionForElems(Domain\&, int, double)}
		\item{std::vector<double,std::allocator<double>>}
		\item{CommSend(Domain\&, int, int, double\& (Domain::**)(int), int, int, int, bool, bool)}
		\item{CommMonoQ(Domain\&)}
		\item{Domain::Domain(int, int, int, int, int, int, int, int, int)}
		\item{CommSBN(Domain\&, int, double\& (Domain::**)(int))}
		\item{CommRecv(Domain\&, int, int, int, int, int, bool, bool)}
		\item{ParseCommandLineOptions(int, char**, int, cmdLineOpts*)}
		\item{InitMeshDecomp(int, int, int*, int*, int*, int*)}
		\item{Domain::~Domain()}
		\item{VerifyAndWriteFinalOutput(double, Domain\&, int, int)}
	\end{enumerate}
\end{enumerate}
\item{Is the measured execution time of the application affected by gprof? Hint: use the time command to determine this.}
The measured execution time does not differ with use of gprof. This was tested with the four benchmarks with the following results:
\begin{center}
\begin{tabular}{|c|c|c|}
\hline
&Time without Gprof (s) & Time with Gprof (s)\\
\hline
Serial &12.881&\\ \hline
OpenMP &20.461&\\ \hline
MPI &118.357&\\ \hline
Hybrid &46.481&\\ \hline
\end{tabular}
\end{center}
\item{Can gprof analyze the loops (for, while, do-while, etc.) of the application?(1 point)}\\
No, gprof cannot be used to analyse loops.
\item{Is gprof capable of analyzing parallel applications?}\\
Yes, gprof can be used to analyze parallel applications.
\item{What is necessary to analyze parallel applications?}\\
Gprof typically only produces one gmon.out output file for the main process. However, in a parallel process, multiplie output files are required. One must update environmental variables such that each thread produces one output file. These files can then be aggregated to determine the overall behavior of all threads together or analyzed separately.
\item{Where there performance differences between the GNU++ and the Intel compiler?}

\end{enumerate}

\subsection{Compiler flags}
Lorem ipsum

\subsection{Optimization pragmas}
Lorem ipsum

\subsection{Inline assembler}
Lorem ipsum

% Figure example
\begin{figure}[h!] % h=here, t=top, b=bottom, p=(extra)page, !=force
 	\begin{center}
 		\includegraphics[width=.9\linewidth]{figure.png} % It searches in the Figures/ folder!
 		\caption{Caption text}
 		\label{fig:figureLabelName}
 	\end{center}
\end{figure}

\section{Name of assignment 2}
\subsection{Name of sub-assignment 2.1}
Lorem ipsum

\end{document}

%eof
