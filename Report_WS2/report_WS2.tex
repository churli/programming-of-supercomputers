% Report
\documentclass{article}

% Here set the various packages
% Packages to load
\usepackage[english]{babel}
% %%% Support some german text
% \usepackage{ngerman}
% \usepackage[latin1]{inputenc}   % für Umlaute
%%%
% \usepackage[utf8]{inputenc}
\usepackage[T1]{fontenc}
\usepackage{microtype}

%%%
% \usepackage[inline]{enumitem} % Required for the "description" list.

%%% Fix for not hyperlinking citations
\makeatletter
\let\NAT@parse\undefined
\makeatother
\usepackage{hyperref}
% 
\usepackage{cite}
% \ifx\pdfoutput\undefined
% 	\usepackage{graphicx}
% \else
% 	\usepackage[pdftex]{graphicx}
% \fi
\usepackage{graphicx}
\graphicspath{{Figures/}}
\usepackage{amsmath}
% \interdisplaylinepenalty=2500

% Shading of questions. Use the "shaded" environment or the "\hl{}" command.
\usepackage{framed}
% \usepackage[dvipsnames]{color}
\usepackage[svgnames]{xcolor}
\usepackage{soul}
% Nice colours: Gainsboro, LightGoldenrod, LightSteelBlue
% furter ref: https://www.latextemplates.com/svgnames-colors
\definecolor{shadecolor}{named}{Gainsboro}
\sethlcolor{Gainsboro}

%%% Todo margin notes (enable/disable)
\usepackage{todonotes}
% \usepackage[disable]{todonotes}
%%%

%%% For more flexible enumerate environments
\usepackage{enumitem}

%%% For allowing verbatim text in captions
\usepackage{cprotect}
% Usage: \cprotect\caption{blablabla \verb!verbatim text!}

%eof

%%%

\title{Programming of Supercomputers\\Worksheet 2}
\author{
	\begin{tabular}{rl}
		Oleksandr Voloshyn& \texttt{<o.voloshyn@tum.de>}\\ 
		Qunsheng Huang& \texttt{<keefe.huang@tum.de>}\\ 
		Tommaso Bianucci& \texttt{<bianucci@in.tum.de>}
	\end{tabular}
}
\date{\today}

\begin{document}

\maketitle
\renewcommand{\abstractname}{Group members's contributions}
\begin{abstract}
	\begin{center}
		\begin{tabular}{rl}
		% Here write the contributions of the members of the group
		Oleksandr Voloshyn:& worked on xx, yy, zz\\
		Qunsheng Huang:& worked on xx, yy, zz\\
		Tommaso Bianucci:& worked on xx, yy, zz
		\end{tabular}
	\end{center}
\end{abstract}

\section{Task 1}
\subsection{Concepts description}
\begin{description}
	\item[Race condition]

	It is a concurrent access to data by two different threads in which the outcome of the program changes depending on the order of the accesses. This leads to non-deterministic behaviour of the program.

	\item[Deadlock]

	A condition in which two or more threads are waiting on each other before continuing execution, therefore blocking each other execution indefinitely.
	
	\item[Heisenbug] 

	A bug which occurrence is influenced by the presence of a debugger\todo{TO BE CONFIRMED}, therefore it does not happen during debugging, making it more difficult to solve.
	
	\item[Cache coherency and false sharing] 

	Two processors, each having its own cache, load the same data from memory to their caches. Then processor 1 changes this data in its own cache. \emph{Cache coherency} mechanisms ensure that this change is propagated also to the cache of processor 2, to avoid it to work on old data which is not valid anymore. \emph{False sharing} is an unwanted side effect of cache coherency mechanisms and occurs when two processors load the same cache line but work on different data within this cache line. Each write operation from one processor will cause the other one to have to reload its cache, even if the change was in data which will not be used by the other processor.\todo{Write this in a better form}
	
	\item[Load imbalance] 

	A load imbalance is a situation in which two processors are given different amount of work to perform concurrently. This is usually a cause of performance degradation, as the less loaded processor will have to idle while waiting for the more loaded one to finish.
	
	\item[Amdahl's law] 

	If a program has a fraction $f$ which is parallelizable and and the remaining $(1-f)$ which is inherently sequential, we can estimate the time it takes to execute it on $N$ processors as:
	\begin{align}
		T(N) =& f \cdot \frac{T(1)}{N} + (1 - f) \cdot T(1)\\
		=& T(1) \cdot (1 - f + \frac{f}{N})
	\end{align}
	Therefore $lim_{N\to\infty} T(N) = T(1) \cdot (1 - f)$. So we can see that performance improvement is limited by the non-parallelizable fraction of the program.

	\item[Parallelization overhead] This is the additional performance cost caused by managing the parallelization, as e.g. the additional time spent in communication, synchronization, creating the new threads.
	
	\item[Floating-point arithmetic challenges] ~
				\begin{description}
			\item[Comparisons] www
			\item[Definition of zero and signed zero] www
			\item[Cancellation or loss of significance] www
			\item[Amplificatio and error propagation] www
		\end{description}
\end{description}

\subsection{Questions}
\begin{enumerate}
	\item \hl{Which of the concepts affect performance but not correctness?}

	www

	\item \hl{Which of the concepts affect the correctness of the application?} ~

	\begin{enumerate}[label=\Alph*]
		\item \hl{Of these, which are exclusive to parallel programming?}

		www

		\item \hl{Of these, which are not exclusive to parallel programming?}

		www
	\end{enumerate}

	\item \hl{Which of them can occur in OpenMP applications?}

	www
	
	\item \hl{Which of them can occur in MPI applications?}

	www
	
	\item \hl{Is cache coherency necessary on MPI applications with a single process and a single thread per rank? Explain.}

	www
	
	\item \hl{Is Amdahl's law applicable to strong scaling applications? Explain.}

	www
	
	\item \hl{Is Amdahl's law applicable to weak scaling applications? Explain.}

	www
	
	\item \hl{Which of these limit the scalability of applications?}

	www
	
\end{enumerate}

% % Figure example
% \begin{figure}[h!] % h=here, t=top, b=bottom, p=(extra)page, !=force
%  	\begin{center}
%  		\includegraphics[width=.9\linewidth]{figure.png} % It searches in the Figures/ folder!
%  		\caption{Caption text}
%  		\label{fig:figureLabelName}
%  	\end{center}
% \end{figure}

\section{Task 2: TotalView GUI}

\subsection{Include a brief description of the following aspects of TotalViews GUI in the report:}
\begin{itemize}
	\item Session Manager
	
	Manages debuggin sessions, see Fig.\ref{fig:sess_manager}. Displays previously configured debugging systems---allowing editing, copying and deleting of preivous settings, and view their respective configuration setups. Availble configurations include type of session (parallel/sequential), parallel implementation (poe-linux/MPICH etc.), number of parallel processes, number of nodes, environmental variables and input to parallel program upon startup.
	\begin{figure}[p] % h=here, t=top, b=bottom, p=(extra)page, !=force
			\includegraphics[width=.7\linewidth]{SessionManager.png}
		\caption{Session manager}
		\label{fig:sess_manager}
	\end{figure}
	\item Root Window
	
	Lists all processes and threads controlled by TotalView, see \ref{fig:root_window}. Allows user to "Dive" into processes easily, which launches a process window for the selected process. Allows grouping of threads/processes for easier debugging.
		\begin{figure}[p] % h=here, t=top, b=bottom, p=(extra)page, !=force
			\includegraphics[width=.7\linewidth]{RootWindow.png}
		\caption{Root Window}
		\label{fig:root_window}
	\end{figure}
	\item Process Window
	
	Window for one specific process or thread, see Fig. \ref{fig:proc_window}. Provides information on state of the process and its individual threads. Information provided lsited in the various panes below.
	\begin{itemize}
		\item Stack Trace Pane
		
		Displays the call stack with any active threads, ie lists the active subroutines of the active thread. Able to move up and down the call stack by clicking on the stack frame (routine name) of interest. Stack Frame pane and Source pane updated for the specific routine when it is selected. Local variables are also available in the Local Variables (VAR) Panel.
		\item Stack Frame Pane
		
		Displays information on the current thread's variables, ie allows users to see the current/stored values in existing variables.
		\item Source Pane
		
		Displays the source code for the main() function of the thread, process or selected routine
		\item Action Points, Processes, Threads Pane
		\begin{itemize}
		\item Process Tab: The processes tab, or Ranks tab for MPI programs, contains a grid. Each block in the grid represents one process. The color of the respective segments indicates the state of the process. 
		\item Threads Tab: The Threads Tab displays information about the state of your threads. Clicking on a thread tells TotalView to shift the focus within the Process Window to that thread.
		\item Actions Points Tab: Action points are specific actions performed when a particular source line is reached. There are four possible actions: breakpoints, barrier points, eval points, and watch points. Totalview assigns unique ID numbers for each action point, these are seen in the actions points tab
		\end{itemize}
	\end{itemize}
	\begin{figure}[p] % h=here, t=top, b=bottom, p=(extra)page, !=force
			\includegraphics[width=.7\linewidth]{ProcessWindow.png}
		\caption{Process Window}
		\label{fig:proc_window}
	\end{figure}
	\item Variable Window
	
	Displays information about a program's objects. Allows user to change the element values or to cast these values (changing how the value is displayed). You can define fields which change how objects are viewed, for example: expression field (views result of a particular expression, ie. viewing a particular element in an array with expression 'val[3]' instead of 'val'.); type field (changing the data type of the particular element, and so on. The user can also 'dive' into more complex elements to observe or change members of structures or elements of arrays.
	
\end{itemize}

\section{Task 3: Debugging with TotalView}
\begin{enumerate}
\item Describe in the report how the above operations are performed in TotalView
	\begin{itemize}
	\item Control Execution
	
	This controls how the user moves through the program. There are a few options available: go, next, step, out, and run to. Go executes the program normally and only stops with the addition of breakpoints. Next runs the one line of code and advances to the next line. Step runs until the next executable statement is reached. Run to allows the user to run normally until a specific selected line. Out is used to execute the return statement and exit a function. All these options are available as clickable buttons along the top of the Process window.
	\item Setting breakpoints
	
	Breakpoints are the simplest form of Action Point availablein TotalView that stop the current threads execution when a specific line is reached. They can be setup prior to running the process by clicking on the specific line number in the Source Pane or right clicking the line number and selecting "Set Breakpoint". The line number will be highlighted and the new breakpoint will be visible in the Action Point tab.
	\item Diving into functions
	
	"Diving" is the term used by TotalView to focus on a function, process or variable. To dive into a function, one can double-click on the function in the Processes tab or navigate to the function in the source view.
	\item View memory (variables and arrays)
	
	Variables and arrays can be accesed in the Variables window.

	\end{itemize}
\item Give a short explanation on why these operations are important in a debugger.
	\begin{itemize}
	\item Control Execution
	
	Control execution is important when debugging because it allow a systematic manner to execute segments of code. For example, if one needs to pinpoint problematic segments of very large code, perhaps a normal "Run" command with certain breakpoints while observing key variables is a good starting point. On the other hand, if one has a short problematic segment of code, repeatedly running Next to examine the effects of each line of code might be necessary to uncover more insidious bugs. The available options allow for the user to move through code in an extremely flexible manner.
	\item Setting breakpoints
		
	Breakpoints are extremely useful when debugging. They cause the program to halt or stop execution at a specific interval for deeper analysis. TotalView also allows for code execution during action points (known as eval points), which is extremely useful for evaluating behavior without having to specifically include calculation of variables of interest inside the examined code. 
	\item Diving into functions
	
	Diving into functions allows for increased granularity when running code. If one function is identified to be problematic, having the diving functionality allows the user to also control the execution inside the problematic function---allowing a more fine-grained view of the problematic function.	
	\item View memory (variables and arrays)
	
	Most of the time, the side-effect of problematic code is incorrect variables or arrays, ie. data is incorrectly calculated. Viewing specific variables or arrays allows the user to more clearly detect when unwanted behavior occurs. This is especially the case when one is able to evaluate internal variables using code fragments (such as identifying the average of an array without explicitly calculating it in the code).

	\end{itemize}

\end{enumerate}


\listoftodos[TODO List (to be removed in final version)]

\end{document}

%eof
